\section{Introduction}

딥러닝 기술의 혁신 이후 4차 산업 혁명이라는 말이 생길 정도로 사회에는 큰 변화가 나타났다. 
오로지 사람의 영역으로 여겨졌던 많은 분야에서 점차 사람의 역할을 컴퓨터가 대체하는 현상이 벌어지고 있다. 또한 지금 이 순간에도 컴퓨터의 활용 영역은 조금씩 넓어져 가고 있다. 
본 프로젝트에서 목표로 하고 있는 자동 채색 기술도 그 중 하나이다. 
2000년대 중반 웹툰이라는 플랫폼이 등장하면서 죽어가던 만화 시장에 새로운 활기가 불어넣어졌다. 
이후 웹툰 시장은 급성장하여 일 수백 만 명의 접속자를 기록하며 새로운 문화 플랫폼으로서의 역할을 하게 되었다. 
웹툰의 제작 과정은 크게 스토리 구성, 스케치 및 채색의 과정으로 나눌 수 있다.
최근에는 단순 작화보다는 스토리와 시나리오가 웹툰의 인기 요소로서 더 중요한 역할을 차지하며, 작가들 또한 이러한 트렌드에 맞추어 가고 있다.
이러한 상황 속에서 기계학습을 통한 자동 채색 프로그램이 개발된다면 웹툰 산업에서 놀라운 생산성 증대를 기대해 볼 수 있다. 
또한 웹툰뿐만 아니라 2D 채색이 필요한 게임, 애니메이션 산업 등에서도 널리 활용될 것으로 기대되고 있다.

본 프로젝트에서는 2014년에 딥러닝의 새로운 지평을 열었던 Generative Adversarial Network (GAN)를 활용하여 자동 스케치 채색 프로그램을 개발하는 것을 주 목적으로 한다. 
기존에는 GAN을 활용하여 흑백 사진을 채색하거나, 사진을 만화 스타일로 만드는 선행 연구들이 진행되어 왔다.
하지만 이를 실생활에 활용하기 위해서는 흑백 사진이 아닌 단순 스케치를 채색하는 기술이 필요하며, 흑백 사진의 경우 명도와 채도 등의 원래 색의 정보를 어느 정도 가지고 있으므로 스케치 채색에 비해 학습이 잘 이루어지는 편이다. 
즉 기존의 GAN을 이용한 채색 연구에 비해 그 난이도가 높고, 예상되는 활용도 또한 높은 프로젝트이다. 
현재 GAN을 기반으로 하는 style2paint 등의 자동 스케치 채색 기술이 알려져 있으며, 3개월간의 본 프로젝트에서는 style2paint를 기반으로 하여 더 나은 채색 정확도를 가지는 새로운 모델을 구성하는 것이 궁극적인 목표이다.

본 보고서에서는 현재 알려져 있는 채색 기술에 대해 소개한 후, 포스터 세션으로 이루어진 중간 발표까지의 진행 상황과 앞으로의 계획에 대해 서술하였다. 프로젝트 진행 상황에 대하여 세부적으로는 학습 네트워크의 구성, 데이터셋의 선별과 처리 과정, 그리고 실제 채색 시도 결과에 대하여 서술하였다.
